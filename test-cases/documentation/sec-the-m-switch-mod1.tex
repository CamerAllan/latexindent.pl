% arara: pdflatex: {shell: yes, files: [latexindent]}
\fancyhead[R]{\bfseries\thepage%
	\tikz[remember picture,overlay] {
		\node at (1,0){\includegraphics{logo}};
	}}
\section{The \texttt{-m} (\texttt{modifylinebreaks}) switch}
 \label{sec:modifylinebreaks}
 All features described in this section will only be relevant if the \texttt{-m} switch is used.

 \startcontents[the-m-switch]
 \printcontents[the-m-switch]{}{0}{}

\yamltitle{modifylinebreaks}*{fields}
	\begin{wrapfigure}[7]{r}[0pt]{8cm}
		\cmhlistingsfromfile[style=modifylinebreaks]*{../defaultSettings.yaml}[MLB-TCB,width=.85\linewidth,before=\centering]{\texttt{modifyLineBreaks}}{lst:modifylinebreaks}
	\end{wrapfigure}
	\makebox[0pt][r]{%
		\raisebox{-\totalheight}[0pt][0pt]{%
			\tikz\node[opacity=1] at (0,0) {\includegraphics[width=4cm]{logo}};}}%	
	As of Version 3.0, \texttt{latexindent.pl} has the \texttt{-m} switch, which permits \texttt{latexindent.pl} to modify line breaks, according to the specifications in the \texttt{modifyLineBreaks} field.
	\emph{The settings
		in this field will only be considered if the \texttt{-m} switch has been used}.
	A snippet of the default settings of this field is shown in \cref{lst:modifylinebreaks}.

	Having read the previous paragraph, it should sound reasonable that, if you call \texttt{latexindent.pl} using the \texttt{-m} switch, then you give it permission to modify line breaks in your file, but let's be clear: 

	\begin{warning} If you call \texttt{latexindent.pl} with the \texttt{-m} switch, then you are giving it permission to modify line breaks.
		By default, the only thing that will happen is that multiple blank lines will be condensed into one blank line; many other settings are possible, discussed next.
	\end{warning}

\yamltitle{preserveBlankLines}{0|1}
	This field is directly related to \emph{poly-switches}, discussed below.
	By default, it is set to \texttt{1}, which means that blank lines will be protected from removal; however, regardless of this setting, multiple blank lines can be condensed if \texttt{condenseMultipleBlankLinesInto} is greater than \texttt{0}, discussed next.

\yamltitle{condenseMultipleBlankLinesInto}*{integer $\geq 0$}
	Assuming that this switch takes an integer value greater than \texttt{0}, \texttt{latexindent.pl} will condense multiple blank lines into the number of blank lines illustrated by this switch.
	As an example, \cref{lst:mlb-bl} shows a sample file with blank lines; upon running \begin{commandshell}
latexindent.pl myfile.tex -m  
\end{commandshell} the output is shown in \cref{lst:mlb-bl-out}; note that the multiple blank lines have been condensed into one blank line, and note also that we have used the \texttt{-m} switch!

	\begin{minipage}{.45\textwidth}
		\cmhlistingsfromfile{demonstrations/mlb1.tex}{\texttt{mlb1.tex}}{lst:mlb-bl}
	\end{minipage}%
	\hfill
	\begin{minipage}{.45\textwidth}
		\cmhlistingsfromfile{demonstrations/mlb1-out.tex}{\texttt{mlb1.tex} out output}{lst:mlb-bl-out}
	\end{minipage}

\yamltitle{textWrapOptions}*{fields}
	When the \texttt{-m} switch is active \texttt{latexindent.pl} has the ability to wrap text using the options%
	\announce{2017-05-27}{textWrapOptions} specified in the \texttt{textWrapOptions} field, see \cref{lst:textWrapOptions}.
	The value of \texttt{columns} specifies the column at which the text should be wrapped.
	By default, the value of \texttt{columns} is \texttt{0}, so \texttt{latexindent.pl} will \emph{not} wrap text; if you change it to a value of \texttt{2} or more, then text will be wrapped after the character in the specified column.

	\cmhlistingsfromfile[style=textWrapOptions]*{../defaultSettings.yaml}[MLB-TCB,width=.85\linewidth,before=\centering]{\texttt{textWrapOptions}}{lst:textWrapOptions}

	For example, consider the file give in \cref{lst:textwrap1}.

	\begin{widepage}
		\cmhlistingsfromfile{demonstrations/textwrap1.tex}{\texttt{textwrap1.tex}}{lst:textwrap1}
	\end{widepage}

	Using the file \texttt{textwrap1.yaml} in \cref{lst:textwrap1-yaml}, and running the command \begin{commandshell}
latexindent.pl -m textwrap1.tex -o textwrap1-mod1.tex -l textwrap1.yaml
\end{commandshell} we obtain the output in \cref{lst:textwrap1-mod1}.

	\begin{minipage}{.45\linewidth}
		\cmhlistingsfromfile{demonstrations/textwrap1-mod1.tex}{\texttt{textwrap1-mod1.tex}}{lst:textwrap1-mod1}
	\end{minipage}
	\hfill
	\begin{minipage}{.45\linewidth}
		\cmhlistingsfromfile{demonstrations/textwrap1.yaml}[MLB-TCB]{\texttt{textwrap1.yaml}}{lst:textwrap1-yaml}
	\end{minipage}

	The text wrapping routine is performed \emph{after} verbatim environments have been stored, so verbatim environments and verbatim commands are exempt from the routine.
	For example, using the file in \cref{lst:textwrap2}, \begin{widepage} \cmhlistingsfromfile{demonstrations/textwrap2.tex}{\texttt{textwrap2.tex}}{lst:textwrap2} \end{widepage} and running the following command and continuing to use \texttt{textwrap1.yaml} from \cref{lst:textwrap1-yaml}, \begin{commandshell}
latexindent.pl -m textwrap2.tex -o textwrap2-mod1.tex -l textwrap1.yaml
\end{commandshell} then the output is as in \cref{lst:textwrap2-mod1}.
	\begin{widepage}
		\cmhlistingsfromfile{demonstrations/textwrap2-mod1.tex}{\texttt{textwrap2-mod1.tex}}{lst:textwrap2-mod1}
	\end{widepage}
	Furthermore, the text wrapping routine is performed after the trailing comments have been stored, and they are also exempt from text wrapping.
	For example, using the file in \cref{lst:textwrap3} \begin{widepage} \cmhlistingsfromfile{demonstrations/textwrap3.tex}{\texttt{textwrap3.tex}}{lst:textwrap3} \end{widepage} and running the following command and continuing to use \texttt{textwrap1.yaml} from \cref{lst:textwrap1-yaml}, \begin{commandshell}
latexindent.pl -m textwrap3.tex -o textwrap3-mod1.tex -l textwrap1.yaml
\end{commandshell} then the output is as in \cref{lst:textwrap3-mod1}.

	\cmhlistingsfromfile{demonstrations/textwrap3-mod1.tex}{\texttt{textwrap3-mod1.tex}}{lst:textwrap3-mod1}

	\begin{wrapfigure}[6]{r}[0pt]{8cm}
		\cmhlistingsfromfile[style=textWrapOptionsAll]*{../defaultSettings.yaml}[MLB-TCB,width=.85\linewidth,before=\centering]{\texttt{textWrapOptions}}{lst:textWrapOptionsAll}
	\end{wrapfigure}
	The text wrapping routine of \texttt{latexindent.pl} is performed by the \texttt{Text::Wrap} module, which provides a \texttt{separator} feature to separate lines with characters other than a new line (see \cite{textwrap}).
	By default, the separator is empty (see \cref{lst:textWrapOptionsAll}) which means that a new line token will be used, but you can change it as you see fit.

	For example starting with the file in \cref{lst:textwrap4} 

	\cmhlistingsfromfile{demonstrations/textwrap4.tex}{\texttt{textwrap4.tex}}{lst:textwrap4} and using \texttt{textwrap2.yaml} from \cref{lst:textwrap2-yaml} with the following command \begin{commandshell}
latexindent.pl -m textwrap4.tex -o textwrap4-mod2.tex -l textwrap2.yaml
\end{commandshell} then we obtain the output in \cref{lst:textwrap4-mod2}.

	\begin{minipage}{.45\linewidth}
		\cmhlistingsfromfile{demonstrations/textwrap4-mod2.tex}{\texttt{textwrap4-mod2.tex}}{lst:textwrap4-mod2}
	\end{minipage}
	\hfill
	\begin{minipage}{.45\linewidth}
		\cmhlistingsfromfile{demonstrations/textwrap2.yaml}[MLB-TCB]{\texttt{textwrap2.yaml}}{lst:textwrap2-yaml}
	\end{minipage}

	\paragraph{Summary of text wrapping}
		It is important to note the following: \begin{itemize} \item Verbatim environments (\vref{lst:verbatimEnvironments}) and verbatim commands (\vref{lst:verbatimCommands}) will \emph{not} be affected by the text wrapping routine (see \vref{lst:textwrap2-mod1});
			\item comments will \emph{not} be affected by the text wrapping routine (see \vref{lst:textwrap3-mod1});
			\item indentation is performed \emph{after} the text wrapping routine; as such, indented code will likely exceed any maximum value set in the \texttt{columns} field.
		\end{itemize}

\subsection{\texttt{oneSentencePerLine}: modifying line breaks for sentences}
	\label{sec:onesentenceperline}
	You can instruct \texttt{latexindent.pl} to format \announce*{2018-01-13}{one sentence per line} your file so that it puts one sentence per line.
	%
	Thank you to \cite{mlep} for helping to shape and test this feature.
	The behaviour of this part of the script is controlled by the switches detailed in \cref{lst:oneSentencePerLine}, all of which we discuss next.

	\cmhlistingsfromfile*[style=oneSentencePerLine]*{../defaultSettings.yaml}[MLB-TCB,width=.85\linewidth,before=\centering]{\texttt{oneSentencePerLine}}{lst:oneSentencePerLine}

\yamltitle{manipulateSentences}{0|1}
	This is a binary switch that details if \texttt{latexindent.pl} should perform the sentence manipulation routine; it is \emph{off} (set to \texttt{0}) by default, and you will need to turn it on (by setting it to \texttt{1}) if you want the script to modify line breaks surrounding and within sentences.

\yamltitle{removeSentenceLineBreaks}{0|1}
	When operating upon sentences \texttt{latexindent.pl} will, by default, remove internal linebreaks as \texttt{removeSentenceLineBreaks} is set to \texttt{1}.
	Setting this switch to \texttt{0} instructs \texttt{latexindent.pl} not to do so.

	For example, consider \texttt{multiple-sentences.tex} shown in \cref{lst:multiple-sentences}.

	\cmhlistingsfromfile*{demonstrations/multiple-sentences.tex}{\texttt{multiple-sentences.tex}}{lst:multiple-sentences}

	If we use the YAML files in \cref{lst:manipulate-sentences-yaml,lst:keep-sen-line-breaks-yaml}, and run the commands \begin{widepage} \begin{commandshell}
latexindent.pl multiple-sentences -m -l=manipulate-sentences.yaml
latexindent.pl multiple-sentences -m -l=keep-sen-line-breaks.yaml
	\end{commandshell} \end{widepage} then we obtain the respective output given in \cref{lst:multiple-sentences-mod1,lst:multiple-sentences-mod2}.

	\begin{minipage}{.5\linewidth}
		\cmhlistingsfromfile*{demonstrations/multiple-sentences-mod1.tex}{\texttt{multiple-sentences.tex} using \cref{lst:manipulate-sentences-yaml}}{lst:multiple-sentences-mod1}
	\end{minipage}
	\hfill
	\begin{minipage}{.5\linewidth}
		\cmhlistingsfromfile*[style=yaml-LST]*{demonstrations/manipulate-sentences.yaml}[MLB-TCB]{\texttt{manipulate-sentences.yaml}}{lst:manipulate-sentences-yaml}
	\end{minipage}

	\begin{minipage}{.5\linewidth}
		\cmhlistingsfromfile*{demonstrations/multiple-sentences-mod2.tex}{\texttt{multiple-sentences.tex} using \cref{lst:keep-sen-line-breaks-yaml}}{lst:multiple-sentences-mod2}
	\end{minipage}
	\hfill
	\begin{minipage}{.5\linewidth}
		\cmhlistingsfromfile*[style=yaml-LST]*{demonstrations/keep-sen-line-breaks.yaml}[MLB-TCB]{\texttt{keep-sen-line-breaks.yaml}}{lst:keep-sen-line-breaks-yaml}
	\end{minipage}

	Notice, in particular, that the `internal' sentence line breaks in \cref{lst:multiple-sentences} have been removed in \cref{lst:multiple-sentences-mod1}, but have not been removed in \cref{lst:multiple-sentences-mod2}.

	The remainder of the settings displayed in \vref{lst:oneSentencePerLine} instruct \texttt{latexindent.pl} on how to define a sentence.
	From the perpesctive of \texttt{latexindent.pl} a sentence must: \begin{itemize} \item \emph{follow} a certain character or set of characters (see \cref{lst:sentencesFollow}); by default, this is either \lstinline!\par!, a blank line, a full stop/period (.)
		      , exclamation mark (!), question mark (?) right brace (\}) or a comment
		      on the previous line;
		\item \emph{begin} with a character type (see \cref{lst:sentencesBeginWith}); by default, this is only capital letters;
		\item \emph{end} with a character (see \cref{lst:sentencesEndWith}); by default, these are
		      full stop/period (.), exclamation mark (!) and question mark (?).
	\end{itemize}
	In each case, you can specify the \texttt{other} field to include any pattern that you would like; you can specify anything in this field using the language of regular expressions.

	\begin{adjustwidth}{-3.5cm}{-2.5cm}
		\begin{minipage}{.36\linewidth}
			\cmhlistingsfromfile*[style=sentencesFollow]*{../defaultSettings.yaml}[MLB-TCB,width=.9\linewidth,before=\centering]{\texttt{sentencesFollow}}{lst:sentencesFollow}
		\end{minipage}
		\hfill
		\begin{minipage}{.31\linewidth}
			\cmhlistingsfromfile*[style=sentencesBeginWith]*{../defaultSettings.yaml}[MLB-TCB,width=.9\linewidth,before=\centering]{\texttt{sentencesBeginWith}}{lst:sentencesBeginWith}
		\end{minipage}
		\hfill
		\begin{minipage}{.31\linewidth}
			\cmhlistingsfromfile*[style=sentencesEndWith]*{../defaultSettings.yaml}[MLB-TCB,width=.9\linewidth,before=\centering]{\texttt{sentencesEndWith}}{lst:sentencesEndWith}
		\end{minipage}
	\end{adjustwidth}

\subsubsection{\texttt{sentencesFollow}}
	Let's explore a few of the switches in \texttt{sentencesFollow}; let's start with \vref{lst:multiple-sentences}, and use the YAML settings given in \cref{lst:sentences-follow1-yaml}.
	Using the command \begin{commandshell}
latexindent.pl multiple-sentences -m -l=sentences-follow1.yaml
\end{commandshell} we obtain the output given in \cref{lst:multiple-sentences-mod3}.

	\begin{minipage}{.5\linewidth}
		\cmhlistingsfromfile*{demonstrations/multiple-sentences-mod3.tex}{\texttt{multiple-sentences.tex} using \cref{lst:sentences-follow1-yaml}}{lst:multiple-sentences-mod3}
	\end{minipage}
	\hfill
	\begin{minipage}{.5\linewidth}
		\cmhlistingsfromfile*[style=yaml-LST]*{demonstrations/sentences-follow1.yaml}[MLB-TCB]{\texttt{sentences-follow1.yaml}}{lst:sentences-follow1-yaml}
	\end{minipage}

	Notice that, because \texttt{blankLine} is set to \texttt{0}, \texttt{latexindent.pl} will not seek sentences following a blank line, and so the fourth sentence has not been accounted for.

	We can explore the \texttt{other} field in \cref{lst:sentencesFollow} with the \texttt{.tex} file detailed in \cref{lst:multiple-sentences1}.

	\cmhlistingsfromfile*{demonstrations/multiple-sentences1.tex}{\texttt{multiple-sentences1.tex}}{lst:multiple-sentences1}

	Upon running the following commands \begin{widepage} \begin{commandshell}
latexindent.pl multiple-sentences1 -m -l=manipulate-sentences.yaml
latexindent.pl multiple-sentences1 -m -l=manipulate-sentences.yaml,sentences-follow2.yaml
	\end{commandshell} \end{widepage} then we obtain the respective output given in \cref{lst:multiple-sentences1-mod1,lst:multiple-sentences1-mod2}.
	\cmhlistingsfromfile*{demonstrations/multiple-sentences1-mod1.tex}{\texttt{multiple-sentences1.tex} using \vref{lst:manipulate-sentences-yaml}}{lst:multiple-sentences1-mod1}

	\begin{minipage}{.55\linewidth}
		\cmhlistingsfromfile*{demonstrations/multiple-sentences1-mod2.tex}{\texttt{multiple-sentences1.tex} using \cref{lst:sentences-follow2-yaml}}{lst:multiple-sentences1-mod2}
	\end{minipage}
	\hfill
	\begin{minipage}{.45\linewidth}
		\cmhlistingsfromfile*[style=yaml-LST]*{demonstrations/sentences-follow2.yaml}[MLB-TCB]{\texttt{sentences-follow2.yaml}}{lst:sentences-follow2-yaml}
	\end{minipage}

	Notice that in \cref{lst:multiple-sentences1-mod1} the first sentence after the \texttt{)} has not been accounted for, but that following the inclusion of \cref{lst:sentences-follow2-yaml}, the output given in \cref{lst:multiple-sentences1-mod2} demonstrates that the sentence \emph{has} been accounted for correctly.

\subsubsection{\texttt{sentencesBeginWith}}
	By default, \texttt{latexindent.pl} will only assume that sentences begin with the upper case letters \texttt{A-Z}; you can instruct the script to define sentences to begin with lower case letters (see \cref{lst:sentencesBeginWith}), and we can use the \texttt{other} field to define sentences to begin with other characters.

	\cmhlistingsfromfile*{demonstrations/multiple-sentences2.tex}{\texttt{multiple-sentences2.tex}}{lst:multiple-sentences2}

	Upon running the following commands \begin{widepage} \begin{commandshell}
latexindent.pl multiple-sentences2 -m -l=manipulate-sentences.yaml
latexindent.pl multiple-sentences2 -m -l=manipulate-sentences.yaml,sentences-begin1.yaml
	\end{commandshell} \end{widepage} then we obtain the respective output given in \cref{lst:multiple-sentences2-mod1,lst:multiple-sentences2-mod2}.
	\cmhlistingsfromfile*{demonstrations/multiple-sentences2-mod1.tex}{\texttt{multiple-sentences2.tex} using \vref{lst:manipulate-sentences-yaml}}{lst:multiple-sentences2-mod1}

	\begin{minipage}{.55\linewidth}
		\cmhlistingsfromfile*{demonstrations/multiple-sentences2-mod2.tex}{\texttt{multiple-sentences2.tex} using \cref{lst:sentences-begin1-yaml}}{lst:multiple-sentences2-mod2}
	\end{minipage}
	\hfill
	\begin{minipage}{.45\linewidth}
		\cmhlistingsfromfile*[style=yaml-LST]*{demonstrations/sentences-begin1.yaml}[MLB-TCB]{\texttt{sentences-begin1.yaml}}{lst:sentences-begin1-yaml}
	\end{minipage}
	Notice that in \cref{lst:multiple-sentences2-mod1}, the first sentence has been accounted for but that the subsequent sentences have not.
	In \cref{lst:multiple-sentences2-mod2}, all of the sentences have been accounted for, because the \texttt{other} field in \cref{lst:sentences-begin1-yaml} has defined sentences to begin with either \lstinline!$! or any numeric digit, \texttt{0} to \texttt{9}.

\subsubsection{\texttt{sentencesEndWith}}
	Let's return to \vref{lst:multiple-sentences}; we have already seen the default way in which \texttt{latexindent.pl} will operate on the sentences in this file in \vref{lst:multiple-sentences-mod1}.
	We can populate the \texttt{other} field with any character that we wish; for example, using the YAML specified in \cref{lst:sentences-end1-yaml} and the command \begin{commandshell}
latexindent.pl multiple-sentences -m -l=sentences-end1.yaml
latexindent.pl multiple-sentences -m -l=sentences-end2.yaml
\end{commandshell} then we obtain the output in \cref{lst:multiple-sentences-mod4}.

	\begin{minipage}{.5\linewidth}
		\cmhlistingsfromfile*{demonstrations/multiple-sentences-mod4.tex}{\texttt{multiple-sentences.tex} using \cref{lst:sentences-end1-yaml}}{lst:multiple-sentences-mod4}
	\end{minipage}
	\hfill
	\begin{minipage}{.5\linewidth}
		\cmhlistingsfromfile*[style=yaml-LST]*{demonstrations/sentences-end1.yaml}[MLB-TCB]{\texttt{sentences-end1.yaml}}{lst:sentences-end1-yaml}
	\end{minipage}

	\begin{minipage}{.5\linewidth}
		\cmhlistingsfromfile*{demonstrations/multiple-sentences-mod5.tex}{\texttt{multiple-sentences.tex} using \cref{lst:sentences-end2-yaml}}{lst:multiple-sentences-mod5}
	\end{minipage}
	\hfill
	\begin{minipage}{.5\linewidth}
		\cmhlistingsfromfile*[style=yaml-LST]*{demonstrations/sentences-end2.yaml}[MLB-TCB]{\texttt{sentences-end2.yaml}}{lst:sentences-end2-yaml}
	\end{minipage}

	There is a subtle difference between the output in \cref{lst:multiple-sentences-mod4,lst:multiple-sentences-mod5}; in particular, in \cref{lst:multiple-sentences-mod4} the word \texttt{sentence} has not been defined as a sentence, because we have not instructed \texttt{latexindent.pl} to begin sentences with lower case letters.
	We have changed this by using the settings in \cref{lst:sentences-end2-yaml}, and the associated output in \cref{lst:multiple-sentences-mod5} reflects this.

	Referencing \vref{lst:sentencesEndWith}, you'll notice that there is a field called \texttt{basicFullStop}, which is set to \texttt{0}, and that the \texttt{betterFullStop} is set to \texttt{1} by default.

	Let's consider the file shown in \cref{lst:url}.

	\cmhlistingsfromfile*{demonstrations/url.tex}{\texttt{url.tex}}{lst:url}

	Upon running the following commands \begin{commandshell}
latexindent.pl url -m -l=manipulate-sentences.yaml
\end{commandshell} we obtain the output given in \cref{lst:url-mod1}.

	\cmhlistingsfromfile*{demonstrations/url-mod1.tex}{\texttt{url.tex} using \vref{lst:manipulate-sentences-yaml}}{lst:url-mod1}

	Notice that the full stop within the url has been interpreted correctly.
	This is because, within the \texttt{betterFullStop}, full stops at the end of sentences have the following properties: \begin{itemize} \item they are ignored within \texttt{e.g.} and \texttt{i.e.};
		\item they can not be immediately followed by a lower case or upper case letter;
		\item they can not be immediately followed by a hyphen, comma, or number.
	\end{itemize}
	If you find that the \texttt{betterFullStop} does not work for your purposes, then you can switch it off by setting it to \texttt{0}, and you can experiment with the \texttt{other} field.

	The \texttt{basicFullStop} routine should probably be avoided in most situations, as it does not accomodate the specifications above.
	For example, using the YAML in \cref{lst:alt-full-stop1-yaml} gives the output from the following command in \cref{lst:url-mod2}.
	\begin{commandshell}
latexindent.pl url -m -l=alt-full-stop1.yaml
\end{commandshell}

	\begin{adjustwidth}{-3.5cm}{-1.5cm}
		\begin{minipage}{.6\linewidth}
			\cmhlistingsfromfile*{demonstrations/url-mod2.tex}{\texttt{url.tex} using \cref{lst:alt-full-stop1-yaml}}{lst:url-mod2}
		\end{minipage}
		\hfill
		\begin{minipage}{.4\linewidth}
			\cmhlistingsfromfile*[style=yaml-LST]*{demonstrations/alt-full-stop1.yaml}[MLB-TCB]{\texttt{alt-full-stop1.yaml}}{lst:alt-full-stop1-yaml}
		\end{minipage}
	\end{adjustwidth}

	Notice that the full stop within the URL has not been accomodated correctly because of the non-default settings in \cref{lst:alt-full-stop1-yaml}.

\subsubsection{Features of the \texttt{oneSentencePerLine} routine}
	The sentence manipulation routine takes place \emph{after} verbatim environments, preamble and trailing comments have been accounted for; this means that any characters within these types of code blocks will not be part of the sentence manipulation routine.

	For example, if we begin with the \texttt{.tex} file in \cref{lst:multiple-sentences3}, and run the command \begin{commandshell}
latexindent.pl multiple-sentences3 -m -l=manipulate-sentences.yaml
	\end{commandshell} then we obtain the output in \cref{lst:multiple-sentences3-mod1}.
	\cmhlistingsfromfile*{demonstrations/multiple-sentences3.tex}{\texttt{multiple-sentences3.tex}}{lst:multiple-sentences3}
	\cmhlistingsfromfile*{demonstrations/multiple-sentences3-mod1.tex}{\texttt{multiple-sentences3.tex} using \vref{lst:manipulate-sentences-yaml}}{lst:multiple-sentences3-mod1}

	Furthermore, if sentences run across environments then, by default, the line breaks internal to the sentence will be removed.
	For example, if we use the \texttt{.tex} file in \cref{lst:multiple-sentences4} and run the commands \begin{commandshell}
latexindent.pl multiple-sentences4 -m -l=manipulate-sentences.yaml
latexindent.pl multiple-sentences4 -m -l=keep-sen-line-breaks.yaml
	\end{commandshell} then we obtain the output in \cref{lst:multiple-sentences4-mod1,lst:multiple-sentences4-mod2}.
	\cmhlistingsfromfile*{demonstrations/multiple-sentences4.tex}{\texttt{multiple-sentences4.tex}}{lst:multiple-sentences4}
	\begin{widepage}
		\cmhlistingsfromfile*{demonstrations/multiple-sentences4-mod1.tex}{\texttt{multiple-sentences4.tex} using \vref{lst:manipulate-sentences-yaml}}{lst:multiple-sentences4-mod1}
	\end{widepage}
	\cmhlistingsfromfile*{demonstrations/multiple-sentences4-mod2.tex}{\texttt{multiple-sentences4.tex} using \vref{lst:keep-sen-line-breaks-yaml}}{lst:multiple-sentences4-mod2}
	Once you've read \cref{sec:poly-switches}, you will know that you can accomodate the removal of internal sentence line breaks by using the YAML in \cref{lst:item-rules2-yaml} and the command \begin{commandshell}
latexindent.pl multiple-sentences4 -m -l=item-rules2.yaml
	\end{commandshell} the output of which is shown in \cref{lst:multiple-sentences4-mod3}.

	\begin{minipage}{.5\linewidth}
		\cmhlistingsfromfile*{demonstrations/multiple-sentences4-mod3.tex}{\texttt{multiple-sentences4.tex} using \cref{lst:item-rules2-yaml}}{lst:multiple-sentences4-mod3}
	\end{minipage}
	\hfill
	\begin{minipage}{.5\linewidth}
		\cmhlistingsfromfile*[style=yaml-LST]*{demonstrations/item-rules2.yaml}[MLB-TCB]{\texttt{item-rules2.yaml}}{lst:item-rules2-yaml}
	\end{minipage}

\subsection{\texttt{removeParagraphLineBreaks}: modifying line breaks for paragraphs}
	When the \texttt{-m} switch is active \texttt{latexindent.pl} has the ability to remove line breaks%
	\announce{2017-05-27}{removeParagraphLineBreaks} from within paragraphs; the behaviour is controlled by the \texttt{removeParagraphLineBreaks} field, detailed in \cref{lst:removeParagraphLineBreaks}.
	Thank you to \cite{jowens} for shaping and assisting with the testing of this feature.
\yamltitle{removeParagraphLineBreaks}*{fields}
	This feature is considered complimentary to the \texttt{oneSentencePerLine} feature described in \vref{sec:onesentenceperline}.

	\cmhlistingsfromfile[style=removeParagraphLineBreaks]*{../defaultSettings.yaml}[MLB-TCB,width=.85\linewidth,before=\centering]{\texttt{removeParagraphLineBreaks}}{lst:removeParagraphLineBreaks}

	This routine can be turned on \emph{globally} for \emph{every} code block type known to \texttt{latexindent.pl} (see \vref{tab:code-blocks}) by using the \texttt{all} switch; by default, this switch is \emph{off}.
	Assuming that the \texttt{all} switch is off, then the routine can be controlled on a per-code-block-type basis, and within that, on a per-name basis.
	We will consider examples of each of these in turn, but before we do, let's specify what \texttt{latexindent.pl} considers as a paragraph: \begin{itemize} \item it must begin on its own line with either an alphabetic or numeric character, and not with any of the code-block types detailed in \vref{tab:code-blocks};
		\item it can include line breaks, but finishes when it meets either a blank line, a \lstinline!\par! command, or any of the user-specified settings in the \texttt{paragraphsStopAt} field, detailed in \vref{lst:paragraphsStopAt}.
	\end{itemize}

	Let's start with the \texttt{.tex} file in \cref{lst:shortlines}, together with the YAML settings in \cref{lst:remove-para1-yaml}.

	\begin{minipage}{.45\linewidth}
		\cmhlistingsfromfile[showspaces=true]{demonstrations/shortlines.tex}{\texttt{shortlines.tex}}{lst:shortlines}
	\end{minipage}
	\hfill
	\begin{minipage}{.49\linewidth}
		\cmhlistingsfromfile{demonstrations/remove-para1.yaml}[MLB-TCB]{\texttt{remove-para1.yaml}}{lst:remove-para1-yaml}
	\end{minipage}

	Upon running the command \begin{commandshell}
latexindent.pl -m shortlines.tex -o shortlines1.tex -l remove-para1.yaml
\end{commandshell} then we obtain the output given in \cref{lst:shortlines1}.

	\cmhlistingsfromfile[showspaces=true]{demonstrations/shortlines1.tex}{\texttt{shortlines1.tex}}{lst:shortlines1}

	Keen readers may notice that some trailing white space must be present in the file in \cref{lst:shortlines} which has crept in to the output in \cref{lst:shortlines1}.
	This can be fixed using the YAML file in \vref{lst:removeTWS-before} and running, for example, \begin{commandshell}
latexindent.pl -m shortlines.tex -o shortlines1-tws.tex -l remove-para1.yaml,removeTWS-before.yaml  
    \end{commandshell} in which case the output is as in \cref{lst:shortlines1-tws}; notice that the double spaces present in \cref{lst:shortlines1} have been addressed.

	\cmhlistingsfromfile[showspaces=true]{demonstrations/shortlines1-tws.tex}{\texttt{shortlines1-tws.tex}}{lst:shortlines1-tws}

	Keeping with the settings in \cref{lst:remove-para1-yaml}, we note that the \texttt{all} switch applies to \emph{all} code block types.
	So, for example, let's consider the files in \cref{lst:shortlines-mand,lst:shortlines-opt} 

	\begin{minipage}{.45\linewidth}
		\cmhlistingsfromfile{demonstrations/shortlines-mand.tex}{\texttt{shortlines-mand.tex}}{lst:shortlines-mand}
	\end{minipage} \hfill
	\begin{minipage}{.45\linewidth}
		\cmhlistingsfromfile{demonstrations/shortlines-opt.tex}{\texttt{shortlines-opt.tex}}{lst:shortlines-opt}
	\end{minipage} 

	Upon running the commands \begin{widepage} \begin{commandshell}
latexindent.pl -m shortlines-mand.tex -o shortlines-mand1.tex -l remove-para1.yaml
latexindent.pl -m shortlines-opt.tex -o shortlines-opt1.tex -l remove-para1.yaml
\end{commandshell} \end{widepage} 

	then we obtain the respective output given in \cref{lst:shortlines-mand1,lst:shortlines-opt1}.

	\cmhlistingsfromfile{demonstrations/shortlines-mand1.tex}{\texttt{shortlines-mand1.tex}}{lst:shortlines-mand1}
	\cmhlistingsfromfile{demonstrations/shortlines-opt1.tex}{\texttt{shortlines-opt1.tex}}{lst:shortlines-opt1}

	Assuming that we turn \emph{off} the \texttt{all} switch (by setting it to \texttt{0}), then we can control the behaviour of \texttt{removeParagraphLineBreaks} either on a per-code-block-type basis, or on a per-name basis.

	For example, let's use the code in \cref{lst:shortlines-envs}, and consider the settings in \cref{lst:remove-para2-yaml,lst:remove-para3-yaml}; note that in \cref{lst:remove-para2-yaml} we specify that \emph{every} environment should receive treatment from the routine, while in \cref{lst:remove-para3-yaml} we specify that \emph{only} the \texttt{one} environment should receive the treatment.

	\begin{minipage}{.45\linewidth}
		\cmhlistingsfromfile{demonstrations/shortlines-envs.tex}{\texttt{shortlines-envs.tex}}{lst:shortlines-envs}
	\end{minipage}
	\hfill
	\begin{minipage}{.49\linewidth}
		\cmhlistingsfromfile{demonstrations/remove-para2.yaml}[MLB-TCB]{\texttt{remove-para2.yaml}}{lst:remove-para2-yaml}
		\cmhlistingsfromfile{demonstrations/remove-para3.yaml}[MLB-TCB]{\texttt{remove-para3.yaml}}{lst:remove-para3-yaml}
	\end{minipage}

	Upon running the commands \begin{widepage} \begin{commandshell}
latexindent.pl -m shortlines-envs.tex -o shortlines-envs2.tex -l remove-para2.yaml
latexindent.pl -m shortlines-envs.tex -o shortlines-envs3.tex -l remove-para3.yaml
\end{commandshell} \end{widepage} then we obtain the respective output given in \cref{lst:shortlines-envs2,lst:shortlines-envs3}.

	\cmhlistingsfromfile{demonstrations/shortlines-envs2.tex}{\texttt{shortlines-envs2.tex}}{lst:shortlines-envs2}
	\cmhlistingsfromfile{demonstrations/shortlines-envs3.tex}{\texttt{shortlines-envs3.tex}}{lst:shortlines-envs3}

	The remaining code-block types can be customized in analogous ways, although note that \texttt{commands}, \texttt{keyEqualsValuesBracesBrackets}, \texttt{namedGroupingBracesBrackets}, \texttt{UnNamedGroupingBracesBrackets} are controlled by the \texttt{optionalArguments} and the \texttt{mandatoryArguments}.

	The only special case is the \texttt{masterDocument} field; this is designed for `chapter'-type files that may contain paragraphs that are not within any other code-blocks.
	For example, consider the file in \cref{lst:shortlines-md}, with the YAML settings in \cref{lst:remove-para4-yaml}.

	\begin{minipage}{.45\linewidth}
		\cmhlistingsfromfile{demonstrations/shortlines-md.tex}{\texttt{shortlines-md.tex}}{lst:shortlines-md}
	\end{minipage}
	\hfill
	\begin{minipage}{.49\linewidth}
		\cmhlistingsfromfile{demonstrations/remove-para4.yaml}[MLB-TCB]{\texttt{remove-para4.yaml}}{lst:remove-para4-yaml}
	\end{minipage}

	Upon running the following command \begin{widepage} \begin{commandshell}
latexindent.pl -m shortlines-md.tex -o shortlines-md4.tex -l remove-para4.yaml
\end{commandshell} \end{widepage} then we obtain the output in \cref{lst:shortlines-md4}.
	\cmhlistingsfromfile{demonstrations/shortlines-md4.tex}{\texttt{shortlines-md4.tex}}{lst:shortlines-md4}

\yamltitle{paragraphsStopAt}*{fields}
	The paragraph line break routine considers blank lines and the \lstinline|\par| command to be the end of a paragraph;%
	\announce{2017-05-27}{paragraphsStopAt} you can fine tune the behaviour of the routine further by using the \texttt{paragraphsStopAt} fields, shown in \cref{lst:paragraphsStopAt}.

	\cmhlistingsfromfile[style=paragraphsStopAt]*{../defaultSettings.yaml}[MLB-TCB,width=.85\linewidth,before=\centering]{\texttt{paragraphsStopAt}}{lst:paragraphsStopAt}

	The fields specified in \texttt{paragraphsStopAt} tell \texttt{latexindent.pl} to stop the current paragraph when it reaches a line that \emph{begins} with any of the code-block types specified as \texttt{1} in \cref{lst:paragraphsStopAt}.
	By default, you'll see that the paragraph line break routine will stop when it reaches an environment at the beginning of a line.
	It is \emph{not} possible to specify these fields on a per-name basis.

	Let's use the \texttt{.tex} file in \cref{lst:sl-stop}; we will, in turn, consider the settings in \cref{lst:stop-command-yaml,lst:stop-comment-yaml}.

	\begin{minipage}{.45\linewidth}
		\cmhlistingsfromfile{demonstrations/sl-stop.tex}{\texttt{sl-stop.tex}}{lst:sl-stop}
	\end{minipage}
	\hfill
	\begin{minipage}{.49\linewidth}
		\cmhlistingsfromfile{demonstrations/stop-command.yaml}[MLB-TCB]{\texttt{stop-command.yaml}}{lst:stop-command-yaml}

		\cmhlistingsfromfile{demonstrations/stop-comment.yaml}[MLB-TCB]{\texttt{stop-comment.yaml}}{lst:stop-comment-yaml}
	\end{minipage}

	Upon using the settings from \vref{lst:remove-para4-yaml} and running the commands \begin{widepage} \begin{commandshell}
latexindent.pl -m sl-stop.tex -o sl-stop4.tex -l remove-para4.yaml
latexindent.pl -m sl-stop.tex -o sl-stop4-command.tex -l=remove-para4.yaml,stop-command.yaml
latexindent.pl -m sl-stop.tex -o sl-stop4-comment.tex -l=remove-para4.yaml,stop-comment.yaml
    \end{commandshell} \end{widepage} we obtain the respective outputs in \crefrange{lst:sl-stop4}{lst:sl-stop4-comment}; notice in particular that: \begin{itemize} \item in \cref{lst:sl-stop4} the paragraph line break routine has included commands and comments;
		\item in \cref{lst:sl-stop4-command} the paragraph line break routine has \emph{stopped} at the \texttt{emph} command, because in \cref{lst:stop-command-yaml} we have specified \texttt{commands} to be \texttt{1}, and \texttt{emph} is at the beginning of a line;
		\item in \cref{lst:sl-stop4-comment} the paragraph line break routine has \emph{stopped} at the comments, because in \cref{lst:stop-comment-yaml} we have specified \texttt{comments} to be \texttt{1}, and the comment is at the beginning of a line.
	\end{itemize}
	In all outputs in \crefrange{lst:sl-stop4}{lst:sl-stop4-comment} we notice that the paragraph line break routine has stopped at \lstinline!\begin{myenv}! because, by default, \texttt{environments} is set to \texttt{1} in \vref{lst:paragraphsStopAt}.

	\cmhlistingsfromfile{demonstrations/sl-stop4.tex}{\texttt{sl-stop4.tex}}{lst:sl-stop4}
	\cmhlistingsfromfile{demonstrations/sl-stop4-command.tex}{\texttt{sl-stop4-command.tex}}{lst:sl-stop4-command}
	\cmhlistingsfromfile{demonstrations/sl-stop4-comment.tex}{\texttt{sl-stop4-comment.tex}}{lst:sl-stop4-comment}

\subsection{Poly-switches}
	\label{sec:poly-switches}
	Every other field in the \texttt{modifyLineBreaks} field uses poly-switches, and can take one of \emph{five}%
	\announce{2017-08-21}*{blank line poly-switch} integer values: \begin{itemize}[font=\bfseries]
		\item[$-1$] \emph{remove mode}: line breaks before or after the \emph{<part of thing>} can be removed (assuming that \texttt{preserveBlankLines} is set to \texttt{0});
		\item[0] \emph{off mode}: line breaks will not be modified for the \emph{<part of thing>} under consideration;
		\item[1] \emph{add mode}: a line break will be added before or after the \emph{<part of thing>} under consideration, assuming that there is not already a line break before or after the \emph{<part of thing>};
		\item[2] \emph{comment then add mode}: a comment symbol will be added, followed by a line break before or after the \emph{<part of thing>} under consideration, assuming that there is not already a comment and line break before or after the \emph{<part of thing>};
		\item[3] \emph{add then blank line mode}%
		      \announce{2017-08-21}{blank line poly-switch}: a line break will be added before or after the \emph{<part of thing>} under consideration, assuming that there is not already a line break before or after the \emph{<part of thing>}, followed by a blank line.
		      %%
	\end{itemize}
	In the above, \emph{<part of thing>} refers to either the \emph{begin statement}, \emph{body} or \emph{end statement} of the code blocks detailed in \vref{tab:code-blocks}.
	All poly-switches are \emph{off} by default; \texttt{latexindent.pl} searches first of all for per-name settings, and then followed by global per-thing settings.

\subsection{modifyLineBreaks for environments}
	\label{sec:modifylinebreaks-environments}
	We start by viewing a snippet of \texttt{defaultSettings.yaml} in \cref{lst:environments-mlb}; note that it contains \emph{global} settings (immediately after the \texttt{environments} field) and that \emph{per-name} settings are also allowed -- in the case of \cref{lst:environments-mlb}, settings for \texttt{equation*} have been specified.
	Note that all poly-switches are \emph{off} by default.

	\cmhlistingsfromfile[style=modifylinebreaksEnv]*{../defaultSettings.yaml}[width=.8\linewidth,before=\centering,MLB-TCB]{\texttt{environments}}{lst:environments-mlb}

	Let's begin with the simple example given in \cref{lst:env-mlb1-tex}; note that we have annotated key parts of the file using $\BeginStartsOnOwnLine$, $\BodyStartsOnOwnLine$, $\EndStartsOnOwnLine$ and $\EndFinishesWithLineBreak$, these will be related to fields specified in \cref{lst:environments-mlb}.

	\begin{cmhlistings}[style=tcblatex,escapeinside={(*@}{@*)}]{\texttt{env-mlb1.tex}}{lst:env-mlb1-tex}
before words(*@$\BeginStartsOnOwnLine$@*) \begin{myenv}(*@$\BodyStartsOnOwnLine$@*)body of myenv(*@$\EndStartsOnOwnLine$@*)\end{myenv}(*@$\EndFinishesWithLineBreak$@*) after words
\end{cmhlistings}

\subsubsection[Adding line breaks: \texttt{BeginStartsOnOwnLine} and \texttt{BodyStartsOnOwnLine}]{Adding line breaks using \texttt{BeginStartsOnOwnLine} and \texttt{BodyStartsOnOwnLine}}
	Let's explore \texttt{BeginStartsOnOwnLine} and \texttt{BodyStartsOnOwnLine} in \cref{lst:env-mlb1,lst:env-mlb2}, and in particular, let's allow each of them in turn to take a value of $1$.

	\begin{minipage}{.45\textwidth}
		\cmhlistingsfromfile[style=yaml-LST]*{demonstrations/env-mlb1.yaml}[MLB-TCB]{\texttt{env-mlb1.yaml}}{lst:env-mlb1}
	\end{minipage}
	\hfill
	\begin{minipage}{.45\textwidth}
		\cmhlistingsfromfile[style=yaml-LST]*{demonstrations/env-mlb2.yaml}[MLB-TCB]{\texttt{env-mlb2.yaml}}{lst:env-mlb2}
	\end{minipage}

	After running the following commands, \begin{commandshell}
latexindent.pl -m env-mlb.tex -l env-mlb1.yaml
latexindent.pl -m env-mlb.tex -l env-mlb2.yaml
\end{commandshell} the output is as in \cref{lst:env-mlb-mod1,lst:env-mlb-mod2} respectively.

	\begin{widepage}
		\begin{minipage}{.56\linewidth}
			\cmhlistingsfromfile{demonstrations/env-mlb-mod1.tex}{\texttt{env-mlb.tex} using \cref{lst:env-mlb1}}{lst:env-mlb-mod1}
		\end{minipage}
		\hfill
		\begin{minipage}{.43\linewidth}
			\cmhlistingsfromfile{demonstrations/env-mlb-mod2.tex}{\texttt{env-mlb.tex} using \cref{lst:env-mlb2}}{lst:env-mlb-mod2}
		\end{minipage}
	\end{widepage}

	There are a couple of points to note: \begin{itemize} \item in \cref{lst:env-mlb-mod1} a line break has been added at the point denoted by $\BeginStartsOnOwnLine$ in \cref{lst:env-mlb1-tex}; no other line breaks have been changed;
		\item in \cref{lst:env-mlb-mod2} a line break has been added at the point denoted by $\BodyStartsOnOwnLine$ in \cref{lst:env-mlb1-tex}; furthermore, note that the \emph{body} of \texttt{myenv} has received the appropriate (default) indentation.
	\end{itemize}

	Let's now change each of the \texttt{1} values in \cref{lst:env-mlb1,lst:env-mlb2} so that they are $2$ and save them into \texttt{env-mlb3.yaml} and \texttt{env-mlb4.yaml} respectively (see \cref{lst:env-mlb3,lst:env-mlb4}).

	\begin{minipage}{.45\textwidth}
		\cmhlistingsfromfile[style=yaml-LST]*{demonstrations/env-mlb3.yaml}[MLB-TCB]{\texttt{env-mlb3.yaml}}{lst:env-mlb3}
	\end{minipage}
	\hfill
	\begin{minipage}{.45\textwidth}
		\cmhlistingsfromfile[style=yaml-LST]*{demonstrations/env-mlb4.yaml}[MLB-TCB]{\texttt{env-mlb4.yaml}}{lst:env-mlb4}
	\end{minipage}

	Upon running  commands analogous to the above, we obtain \cref{lst:env-mlb-mod3,lst:env-mlb-mod4}.

	\begin{widepage}
		\begin{minipage}{.56\linewidth}
			\cmhlistingsfromfile{demonstrations/env-mlb-mod3.tex}{\texttt{env-mlb.tex} using \cref{lst:env-mlb3}}{lst:env-mlb-mod3}
		\end{minipage}
		\hfill
		\begin{minipage}{.43\linewidth}
			\cmhlistingsfromfile{demonstrations/env-mlb-mod4.tex}{\texttt{env-mlb.tex} using \cref{lst:env-mlb4}}{lst:env-mlb-mod4}
		\end{minipage}
	\end{widepage}

	Note that line breaks have been added as in \cref{lst:env-mlb-mod1,lst:env-mlb-mod2}, but this time a comment symbol has been added before adding the line break; in both cases, trailing horizontal space has been stripped before doing so.

	Let's%
	\announce{2017-08-21}{demonstration of blank line poly-switch (3)}  now change each of the \texttt{1} values in \cref{lst:env-mlb1,lst:env-mlb2} so that they are $3$ and save them into \texttt{env-mlb5.yaml} and \texttt{env-mlb6.yaml} respectively (see \cref{lst:env-mlb5,lst:env-mlb6}).
	%

	\begin{minipage}{.45\textwidth}
		\cmhlistingsfromfile[style=yaml-LST]*{demonstrations/env-mlb5.yaml}[MLB-TCB]{\texttt{env-mlb5.yaml}}{lst:env-mlb5}
	\end{minipage}
	\hfill
	\begin{minipage}{.45\textwidth}
		\cmhlistingsfromfile[style=yaml-LST]*{demonstrations/env-mlb6.yaml}[MLB-TCB]{\texttt{env-mlb6.yaml}}{lst:env-mlb6}
	\end{minipage}

	Upon running  commands analogous to the above, we obtain \cref{lst:env-mlb-mod5,lst:env-mlb-mod6}.

	\begin{widepage}
		\begin{minipage}{.56\linewidth}
			\cmhlistingsfromfile{demonstrations/env-mlb-mod5.tex}{\texttt{env-mlb.tex} using \cref{lst:env-mlb5}}{lst:env-mlb-mod5}
		\end{minipage}
		\hfill
		\begin{minipage}{.43\linewidth}
			\cmhlistingsfromfile{demonstrations/env-mlb-mod6.tex}{\texttt{env-mlb.tex} using \cref{lst:env-mlb6}}{lst:env-mlb-mod6}
		\end{minipage}
	\end{widepage}

	Note that line breaks have been added as in \cref{lst:env-mlb-mod1,lst:env-mlb-mod2}, but this time a \emph{blank line} has been added after adding the line break.

\subsubsection[Adding line breaks: \texttt{EndStartsOnOwnLine} and \texttt{EndFinishesWithLineBreak}]{Adding line breaks using \texttt{EndStartsOnOwnLine} and \texttt{EndFinishesWithLineBreak}}
	Let's explore \texttt{EndStartsOnOwnLine} and \texttt{EndFinishesWithLineBreak} in \cref{lst:env-mlb7,lst:env-mlb8}, and in particular, let's allow each of them in turn to take a value of $1$.

	\begin{minipage}{.49\textwidth}
		\cmhlistingsfromfile[style=yaml-LST]*{demonstrations/env-mlb7.yaml}[MLB-TCB]{\texttt{env-mlb7.yaml}}{lst:env-mlb7}
	\end{minipage}
	\hfill
	\begin{minipage}{.49\textwidth}
		\cmhlistingsfromfile[style=yaml-LST]*{demonstrations/env-mlb8.yaml}[MLB-TCB]{\texttt{env-mlb8.yaml}}{lst:env-mlb8}
	\end{minipage}

	After running the following commands, \begin{commandshell}
latexindent.pl -m env-mlb.tex -l env-mlb7.yaml
latexindent.pl -m env-mlb.tex -l env-mlb8.yaml
\end{commandshell} the output is as in \cref{lst:env-mlb-mod7,lst:env-mlb-mod8}.

	\begin{widepage}
		\begin{minipage}{.42\linewidth}
			\cmhlistingsfromfile{demonstrations/env-mlb-mod7.tex}{\texttt{env-mlb.tex} using \cref{lst:env-mlb7}}{lst:env-mlb-mod7}
		\end{minipage}
		\hfill
		\begin{minipage}{.57\linewidth}
			\cmhlistingsfromfile{demonstrations/env-mlb-mod8.tex}{\texttt{env-mlb.tex} using \cref{lst:env-mlb8}}{lst:env-mlb-mod8}
		\end{minipage}
	\end{widepage}

	There are a couple of points to note: \begin{itemize} \item in \cref{lst:env-mlb-mod7} a line break has been added at the point denoted by $\EndStartsOnOwnLine$ in \vref{lst:env-mlb1-tex}; no other line breaks have been changed and the \lstinline!\end{myenv}! statement has \emph{not} received indentation (as intended);
		\item in \cref{lst:env-mlb-mod8} a line break has been added at the point denoted by $\EndFinishesWithLineBreak$ in \vref{lst:env-mlb1-tex}.
	\end{itemize}

	Let's now change each of the \texttt{1} values in \cref{lst:env-mlb7,lst:env-mlb8} so that they are $2$ and save them into \texttt{env-mlb9.yaml} and \texttt{env-mlb10.yaml} respectively (see \cref{lst:env-mlb9,lst:env-mlb10}).

	\begin{minipage}{.49\textwidth}
		\cmhlistingsfromfile[style=yaml-LST]*{demonstrations/env-mlb9.yaml}[MLB-TCB]{\texttt{env-mlb9.yaml}}{lst:env-mlb9}
	\end{minipage}
	\hfill
	\begin{minipage}{.49\textwidth}
		\cmhlistingsfromfile[style=yaml-LST]*{demonstrations/env-mlb10.yaml}[MLB-TCB]{\texttt{env-mlb10.yaml}}{lst:env-mlb10}
	\end{minipage}

	Upon running  commands analogous to the above, we obtain \cref{lst:env-mlb-mod9,lst:env-mlb-mod10}.

	\begin{widepage}
		\begin{minipage}{.43\linewidth}
			\cmhlistingsfromfile{demonstrations/env-mlb-mod9.tex}{\texttt{env-mlb.tex} using \cref{lst:env-mlb9}}{lst:env-mlb-mod9}
		\end{minipage}
		\hfill
		\begin{minipage}{.56\linewidth}
			\cmhlistingsfromfile{demonstrations/env-mlb-mod10.tex}{\texttt{env-mlb.tex} using \cref{lst:env-mlb10}}{lst:env-mlb-mod10}
		\end{minipage}
	\end{widepage}

	Note that line breaks have been added as in \cref{lst:env-mlb-mod7,lst:env-mlb-mod8}, but this time a comment symbol has been added before adding the line break; in both cases, trailing horizontal space has been stripped before doing so.

	Let's%
	\announce{2017-08-21}{demonstration of blank line poly-switch (3)} now change each of the \texttt{1} values in \cref{lst:env-mlb7,lst:env-mlb8} so that they are $3$ and save them into \texttt{env-mlb11.yaml} and \texttt{env-mlb12.yaml} respectively (see \cref{lst:env-mlb11,lst:env-mlb12}).
	%

	\begin{minipage}{.49\textwidth}
		\cmhlistingsfromfile[style=yaml-LST]*{demonstrations/env-mlb11.yaml}[MLB-TCB]{\texttt{env-mlb11.yaml}}{lst:env-mlb11}
	\end{minipage}
	\hfill
	\begin{minipage}{.49\textwidth}
		\cmhlistingsfromfile[style=yaml-LST]*{demonstrations/env-mlb12.yaml}[MLB-TCB]{\texttt{env-mlb12.yaml}}{lst:env-mlb12}
	\end{minipage}

	Upon running  commands analogous to the above, we obtain \cref{lst:env-mlb-mod11,lst:env-mlb-mod12}.

	\begin{widepage}
		\begin{minipage}{.42\linewidth}
			\cmhlistingsfromfile{demonstrations/env-mlb-mod11.tex}{\texttt{env-mlb.tex} using \cref{lst:env-mlb11}}{lst:env-mlb-mod11}
		\end{minipage}
		\hfill
		\begin{minipage}{.57\linewidth}
			\cmhlistingsfromfile{demonstrations/env-mlb-mod12.tex}{\texttt{env-mlb.tex} using \cref{lst:env-mlb12}}{lst:env-mlb-mod12}
		\end{minipage}
	\end{widepage}

	Note that line breaks have been added as in \cref{lst:env-mlb-mod7,lst:env-mlb-mod8}, and that a \emph{blank line} has been added after the line break.

\subsubsection{poly-switches only add line breaks when necessary}
	If you ask \texttt{latexindent.pl} to add a line break (possibly with a comment) using a poly-switch value of $1$ (or $2$), it will only do so if necessary.
	For example, if you process the file in \vref{lst:mlb2} using any of the YAML files presented so far in this section, it will be left unchanged.

	\begin{minipage}{.45\linewidth}
		\cmhlistingsfromfile{demonstrations/env-mlb2.tex}{\texttt{env-mlb2.tex}}{lst:mlb2}
	\end{minipage}
	\hfill
	\begin{minipage}{.45\linewidth}
		\cmhlistingsfromfile{demonstrations/env-mlb3.tex}{\texttt{env-mlb3.tex}}{lst:mlb3}
	\end{minipage}

	In contrast, the output from processing the file in \cref{lst:mlb3} will vary depending on the poly-switches used; in \cref{lst:env-mlb3-mod2} you'll see that the comment symbol after the \lstinline!\begin{myenv}! has been moved to the next line, as \texttt{BodyStartsOnOwnLine} is set to \texttt{1}.
	In \cref{lst:env-mlb3-mod4} you'll see that the comment has been accounted for correctly because \texttt{BodyStartsOnOwnLine} has been set to \texttt{2}, and the comment symbol has \emph{not} been moved to its own line.
	You're encouraged to experiment with \cref{lst:mlb3} and by setting the other poly-switches considered so far to \texttt{2} in turn.

	\begin{minipage}{.45\linewidth}
		\cmhlistingsfromfile{demonstrations/env-mlb3-mod2.tex}{\texttt{env-mlb3.tex} using \vref{lst:env-mlb2}}{lst:env-mlb3-mod2}
	\end{minipage}
	\hfill
	\begin{minipage}{.45\linewidth}
		\cmhlistingsfromfile{demonstrations/env-mlb3-mod4.tex}{\texttt{env-mlb3.tex} using \vref{lst:env-mlb4}}{lst:env-mlb3-mod4}
	\end{minipage}

	The details of the discussion in this section have concerned \emph{global} poly-switches in the \texttt{environments} field; each switch can also be specified on a \emph{per-name} basis, which would take priority over the global values; with reference to \vref{lst:environments-mlb}, an example is shown for the \texttt{equation*} environment.

\subsubsection{Removing line breaks (poly-switches set to $-1$)}
	Setting poly-switches to $-1$ tells \texttt{latexindent.pl} to remove line breaks of the \emph{<part of the thing>}, if necessary.
	We will consider the example code given in \cref{lst:mlb4}, noting in particular the positions of the line break highlighters, $\BeginStartsOnOwnLine$, $\BodyStartsOnOwnLine$, $\EndStartsOnOwnLine$ and $\EndFinishesWithLineBreak$, together with the associated YAML files in \crefrange{lst:env-mlb13}{lst:env-mlb16}.

	\begin{minipage}{.45\linewidth}
		\begin{cmhlistings}[style=tcblatex,escapeinside={(*@}{@*)}]{\texttt{env-mlb4.tex}}{lst:mlb4}
before words(*@$\BeginStartsOnOwnLine$@*)
\begin{myenv}(*@$\BodyStartsOnOwnLine$@*)
body of myenv(*@$\EndStartsOnOwnLine$@*)
\end{myenv}(*@$\EndFinishesWithLineBreak$@*)
after words
\end{cmhlistings}
	\end{minipage}%
	\hfill
	\begin{minipage}{.51\textwidth}
		\cmhlistingsfromfile[style=yaml-LST]*{demonstrations/env-mlb13.yaml}[MLB-TCB]{\texttt{env-mlb13.yaml}}{lst:env-mlb13}

		\cmhlistingsfromfile[style=yaml-LST]*{demonstrations/env-mlb14.yaml}[MLB-TCB]{\texttt{env-mlb14.yaml}}{lst:env-mlb14}

		\cmhlistingsfromfile[style=yaml-LST]*{demonstrations/env-mlb15.yaml}[MLB-TCB]{\texttt{env-mlb15.yaml}}{lst:env-mlb15}

		\cmhlistingsfromfile[style=yaml-LST]*{demonstrations/env-mlb16.yaml}[MLB-TCB]{\texttt{env-mlb16.yaml}}{lst:env-mlb16}
	\end{minipage}

	After running the commands \begin{commandshell}
latexindent.pl -m env-mlb4.tex -l env-mlb13.yaml
latexindent.pl -m env-mlb4.tex -l env-mlb14.yaml
latexindent.pl -m env-mlb4.tex -l env-mlb15.yaml
latexindent.pl -m env-mlb4.tex -l env-mlb16.yaml
\end{commandshell} 

	we obtain the respective output in \crefrange{lst:env-mlb4-mod13}{lst:env-mlb4-mod16}.

	\begin{minipage}{.45\linewidth}
		\cmhlistingsfromfile{demonstrations/env-mlb4-mod13.tex}{\texttt{env-mlb4.tex} using \cref{lst:env-mlb13}}{lst:env-mlb4-mod13}
	\end{minipage}
	\hfill
	\begin{minipage}{.45\linewidth}
		\cmhlistingsfromfile{demonstrations/env-mlb4-mod14.tex}{\texttt{env-mlb4.tex} using \cref{lst:env-mlb14}}{lst:env-mlb4-mod14}
	\end{minipage}

	\begin{minipage}{.45\linewidth}
		\cmhlistingsfromfile{demonstrations/env-mlb4-mod15.tex}{\texttt{env-mlb4.tex} using \cref{lst:env-mlb15}}{lst:env-mlb4-mod15}
	\end{minipage}
	\hfill
	\begin{minipage}{.45\linewidth}
		\cmhlistingsfromfile{demonstrations/env-mlb4-mod16.tex}{\texttt{env-mlb4.tex} using \cref{lst:env-mlb16}}{lst:env-mlb4-mod16}
	\end{minipage}

	Notice that in: \begin{itemize} \item \cref{lst:env-mlb4-mod13} the line break denoted by $\BeginStartsOnOwnLine$ in \cref{lst:mlb4} has been removed;
		\item \cref{lst:env-mlb4-mod14} the line break denoted by $\BodyStartsOnOwnLine$ in \cref{lst:mlb4} has been removed;
		\item \cref{lst:env-mlb4-mod15} the line break denoted by $\EndStartsOnOwnLine$ in \cref{lst:mlb4} has been removed;
		\item \cref{lst:env-mlb4-mod16} the line break denoted by $\EndFinishesWithLineBreak$ in \cref{lst:mlb4} has been removed.
	\end{itemize}
	We examined each of these cases separately for clarity of explanation, but you can combine all of the YAML settings in \crefrange{lst:env-mlb13}{lst:env-mlb16} into one file; alternatively, you could tell \texttt{latexindent.pl} to load them all by using the following command, for example \begin{widepage} \begin{commandshell}
latexindent.pl -m env-mlb4.tex -l env-mlb13.yaml,env-mlb14.yaml,env-mlb15.yaml,env-mlb16.yaml
\end{commandshell} \end{widepage} which gives the output in \vref{lst:env-mlb1-tex}.

\subsubsection{About trailing horizontal space}
	Recall that on \cpageref{yaml:removeTrailingWhitespace} we discussed the YAML field \texttt{removeTrailingWhitespace}, and that it has two (binary) switches to determine if horizontal space should be removed \texttt{beforeProcessing} and \texttt{afterProcessing}.
	The \texttt{beforeProcessing} is particularly relevant when considering the \texttt{-m} switch; let's consider the file shown in \cref{lst:mlb5}, which highlights trailing spaces.

	\begin{minipage}{.45\linewidth}
		\begin{cmhlistings}[style=tcblatex,showspaces=true,escapeinside={(*@}{@*)}]{\texttt{env-mlb5.tex}}{lst:mlb5}
before words   (*@$\BeginStartsOnOwnLine$@*) 
\begin{myenv}           (*@$\BodyStartsOnOwnLine$@*)
body of myenv      (*@$\EndStartsOnOwnLine$@*) 
\end{myenv}     (*@$\EndFinishesWithLineBreak$@*)
after words
\end{cmhlistings}
	\end{minipage}
	\hfill
	\begin{minipage}{.45\linewidth}
		\cmhlistingsfromfile[style=yaml-LST]*{demonstrations/removeTWS-before.yaml}[yaml-TCB]{\texttt{removeTWS-before.yaml}}{lst:removeTWS-before}
	\end{minipage}

	The output from the following commands \begin{widepage} \begin{commandshell}
latexindent.pl -m env-mlb5.tex -l env-mlb13.yaml,env-mlb14.yaml,env-mlb15.yaml,env-mlb16.yaml
latexindent.pl -m env-mlb5.tex -l env-mlb13.yaml,env-mlb14.yaml,env-mlb15.yaml,env-mlb16.yaml,removeTWS-before.yaml
\end{commandshell} \end{widepage} is shown, respectively, in \cref{lst:env-mlb5-modAll,lst:env-mlb5-modAll-remove-WS}; note that the trailing horizontal white space has been preserved (by default) in \cref{lst:env-mlb5-modAll}, while in \cref{lst:env-mlb5-modAll-remove-WS}, it has been removed using the switch specified in \cref{lst:removeTWS-before}.

	\begin{widepage}
		\cmhlistingsfromfile[showspaces=true]{demonstrations/env-mlb5-modAll.tex}{\texttt{env-mlb5.tex} using \crefrange{lst:env-mlb4-mod13}{lst:env-mlb4-mod16}}{lst:env-mlb5-modAll}

		\cmhlistingsfromfile[showspaces=true]{demonstrations/env-mlb5-modAll-remove-WS.tex}{\texttt{env-mlb5.tex} using \crefrange{lst:env-mlb4-mod13}{lst:env-mlb4-mod16} \emph{and} \cref{lst:removeTWS-before}}{lst:env-mlb5-modAll-remove-WS}
	\end{widepage}

\subsubsection{poly-switch line break removal and blank lines}
	Now let's consider the file in \cref{lst:mlb6}, which contains blank lines.

	\begin{minipage}{.45\linewidth}
		\begin{cmhlistings}[style=tcblatex,escapeinside={(*@}{@*)}]{\texttt{env-mlb6.tex}}{lst:mlb6}
before words(*@$\BeginStartsOnOwnLine$@*)


\begin{myenv}(*@$\BodyStartsOnOwnLine$@*)


body of myenv(*@$\EndStartsOnOwnLine$@*)


\end{myenv}(*@$\EndFinishesWithLineBreak$@*)

after words
\end{cmhlistings}
	\end{minipage}%
	\hfill
	\begin{minipage}{.45\linewidth}
		\cmhlistingsfromfile[style=yaml-LST]*{demonstrations/UnpreserveBlankLines.yaml}[MLB-TCB]{\texttt{UnpreserveBlankLines.yaml}}{lst:UnpreserveBlankLines}
	\end{minipage}

	Upon running the following commands \begin{widepage} \begin{commandshell}
latexindent.pl -m env-mlb6.tex -l env-mlb13.yaml,env-mlb14.yaml,env-mlb15.yaml,env-mlb16.yaml
latexindent.pl -m env-mlb6.tex -l env-mlb13.yaml,env-mlb14.yaml,env-mlb15.yaml,env-mlb16.yaml,UnpreserveBlankLines.yaml
\end{commandshell} \end{widepage} we receive the respective outputs in \cref{lst:env-mlb6-modAll,lst:env-mlb6-modAll-un-Preserve-Blank-Lines}.
	In \cref{lst:env-mlb6-modAll} we see that the multiple blank lines have each been condensed into one blank line, but that blank lines have \emph{not} been removed by the poly-switches -- this is because, by default, \texttt{preserveBlankLines} is set to \texttt{1}.
	By contrast, in \cref{lst:env-mlb6-modAll-un-Preserve-Blank-Lines}, we have allowed the poly-switches to remove blank lines because, in \cref{lst:UnpreserveBlankLines}, we have set \texttt{preserveBlankLines} to \texttt{0}.

	\begin{widepage}
		\begin{minipage}{.30\linewidth}
			\cmhlistingsfromfile{demonstrations/env-mlb6-modAll.tex}{\texttt{env-mlb6.tex} using \crefrange{lst:env-mlb4-mod13}{lst:env-mlb4-mod16}}{lst:env-mlb6-modAll}
		\end{minipage}
		\hfill
		\begin{minipage}{.65\linewidth}
			\cmhlistingsfromfile{demonstrations/env-mlb6-modAll-un-Preserve-Blank-Lines.tex}{\texttt{env-mlb6.tex} using \crefrange{lst:env-mlb4-mod13}{lst:env-mlb4-mod16} \emph{and} \cref{lst:UnpreserveBlankLines}}{lst:env-mlb6-modAll-un-Preserve-Blank-Lines}
		\end{minipage}
	\end{widepage}

	We can explore this further using the blank-line poly-switch value of $3$; let's use the file given in \cref{lst:env-mlb7-tex}.

	\cmhlistingsfromfile{demonstrations/env-mlb7.tex}{\texttt{env-mlb7.tex}}{lst:env-mlb7-tex}

	Upon running the following commands \begin{commandshell}
latexindent.pl -m env-mlb7.tex -l env-mlb12.yaml,env-mlb13.yaml
latexindent.pl -m env-mlb7.tex -l env-mlb13.yaml,env-mlb14.yaml,UnpreserveBlankLines.yaml
            \end{commandshell} we receive the outputs given in \cref{lst:env-mlb7-preserve,lst:env-mlb7-no-preserve}.

	\cmhlistingsfromfile{demonstrations/env-mlb7-preserve.tex}{\texttt{env-mlb7-preserve.tex}}{lst:env-mlb7-preserve}
	\cmhlistingsfromfile{demonstrations/env-mlb7-no-preserve.tex}{\texttt{env-mlb7-no-preserve.tex}}{lst:env-mlb7-no-preserve}

	Notice that in: \begin{itemize} \item \cref{lst:env-mlb7-preserve} that \lstinline!\end{one}! has added a blank line, because of the value of \texttt{EndFinishesWithLineBreak} in \vref{lst:env-mlb12}, and even though the line break ahead of \lstinline!\begin{two}! should have been removed (because of \texttt{BeginStartsOnOwnLine} in \vref{lst:env-mlb13}), the blank line has been preserved by default;
		\item \cref{lst:env-mlb7-no-preserve}, by contrast, has had the additional line-break removed, because of the settings in \cref{lst:UnpreserveBlankLines}.
	\end{itemize}

\subsection{Poly-switches for other code blocks}
	Rather than repeat the examples shown for the environment code blocks (in \vref{sec:modifylinebreaks-environments}), we choose to detail the poly-switches for all other code blocks in \cref{tab:poly-switch-mapping}; note that each and every one of these poly-switches is \emph{off by default}, i.e, set to \texttt{0}.
	Note also that, by design, line breaks involving \texttt{verbatim}, \texttt{filecontents} and `comment-marked' code blocks (\vref{lst:alignmentmarkup}) can \emph{not} be modified using \texttt{latexindent.pl}.

	\begin{longtable}{m{.2\textwidth}@{\hspace{.75cm}}m{.35\textwidth}@{}m{.4\textwidth}}
		\caption{Poly-switch mappings for all code-block types}\label{tab:poly-switch-mapping} \\
		\toprule
		Code block                             & Sample & Poly-switch mapping                  \\
		\midrule
		environment                            &
		\begin{lstlisting}[style=tcblatex,escapeinside={(*@}{@*)},nolol=true]
before words(*@$\BeginStartsOnOwnLine$@*)
\begin{myenv}(*@$\BodyStartsOnOwnLine$@*)
body of myenv(*@$\EndStartsOnOwnLine$@*)
\end{myenv}(*@$\EndFinishesWithLineBreak$@*)
after words
  \end{lstlisting}
		                                       &
		\begin{tabular}[t]{c@{~}l@{}}
			$\BeginStartsOnOwnLine$     & BeginStartsOnOwnLine     \\
			$\BodyStartsOnOwnLine$      & BodyStartsOnOwnLine      \\
			$\EndStartsOnOwnLine$       & EndStartsOnOwnLine       \\
			$\EndFinishesWithLineBreak$ & EndFinishesWithLineBreak \\
		\end{tabular}
		\\
		\cmidrule{2-3}
		ifelsefi                               &
		\begin{lstlisting}[style=tcblatex,escapeinside={(*@}{@*)},nolol=true]
before words(*@$\BeginStartsOnOwnLine$@*)
\if...(*@$\BodyStartsOnOwnLine$@*)
body of if statement(*@$\ElseStartsOnOwnLine$@*)
\else(*@$\ElseFinishesWithLineBreak$@*)
body of else statement(*@$\EndStartsOnOwnLine$@*)
\fi(*@$\EndFinishesWithLineBreak$@*)
after words
  \end{lstlisting}
		                                       &
		\begin{tabular}[t]{c@{~}l@{}}
			$\BeginStartsOnOwnLine$      & IfStartsOnOwnLine         \\
			$\BodyStartsOnOwnLine$       & BodyStartsOnOwnLine       \\
			$\ElseStartsOnOwnLine$       & ElseStartsOnOwnLine       \\
			$\ElseFinishesWithLineBreak$ & ElseFinishesWithLineBreak \\
			$\EndStartsOnOwnLine$        & FiStartsOnOwnLine         \\
			$\EndFinishesWithLineBreak$  & FiFinishesWithLineBreak   \\
		\end{tabular}
		\\
		\cmidrule{2-3}
		optionalArguments                      &
		\begin{lstlisting}[style=tcblatex,escapeinside={(*@}{@*)},nolol=true]
...(*@$\BeginStartsOnOwnLine$@*)
[(*@$\BodyStartsOnOwnLine$@*)
body of opt arg(*@$\EndStartsOnOwnLine$@*)
](*@$\EndFinishesWithLineBreak$@*)
...
  \end{lstlisting}
		                                       &
		\begin{tabular}[t]{c@{~}l@{}}
			$\BeginStartsOnOwnLine$     & LSqBStartsOnOwnLine\footnote{LSqB stands for Left Square Bracket} \\
			$\BodyStartsOnOwnLine$      & OptArgBodyStartsOnOwnLine                                         \\
			$\EndStartsOnOwnLine$       & RSqBStartsOnOwnLine                                               \\
			$\EndFinishesWithLineBreak$ & RSqBFinishesWithLineBreak                                         \\
		\end{tabular}
		\\
		\cmidrule{2-3}
		mandatoryArguments                     &
		\begin{lstlisting}[style=tcblatex,escapeinside={(*@}{@*)},nolol=true]
...(*@$\BeginStartsOnOwnLine$@*)
{(*@$\BodyStartsOnOwnLine$@*)
body of mand arg(*@$\EndStartsOnOwnLine$@*)
}(*@$\EndFinishesWithLineBreak$@*)
...
  \end{lstlisting}
		                                       &
		\begin{tabular}[t]{c@{~}l@{}}
			$\BeginStartsOnOwnLine$     & LCuBStartsOnOwnLine\footnote{LCuB stands for Left Curly Brace} \\
			$\BodyStartsOnOwnLine$      & MandArgBodyStartsOnOwnLine                                     \\
			$\EndStartsOnOwnLine$       & RCuBStartsOnOwnLine                                            \\
			$\EndFinishesWithLineBreak$ & RCuBFinishesWithLineBreak                                      \\
		\end{tabular}
		\\
		\cmidrule{2-3}
		commands                               &
		\begin{lstlisting}[style=tcblatex,escapeinside={(*@}{@*)},morekeywords={mycommand},nolol=true,]
before words(*@$\BeginStartsOnOwnLine$@*)
\mycommand(*@$\BodyStartsOnOwnLine$@*)
(*@$\langle$\itshape{arguments}$\rangle$@*)
  \end{lstlisting}
		                                       &
		\begin{tabular}[t]{c@{~}l@{}}
			$\BeginStartsOnOwnLine$ & CommandStartsOnOwnLine           \\
			$\BodyStartsOnOwnLine$  & CommandNameFinishesWithLineBreak \\
		\end{tabular}
		\\
		\cmidrule{2-3}
		namedGroupingBraces Brackets           &
		\begin{lstlisting}[style=tcblatex,escapeinside={(*@}{@*)},morekeywords={myname},nolol=true,]
before words(*@$\BeginStartsOnOwnLine$@*)
myname(*@$\BodyStartsOnOwnLine$@*)
(*@$\langle$\itshape{braces/brackets}$\rangle$@*)
  \end{lstlisting}
		                                       &
		\begin{tabular}[t]{c@{~}l@{}}
			$\BeginStartsOnOwnLine$ & NameStartsOnOwnLine       \\
			$\BodyStartsOnOwnLine$  & NameFinishesWithLineBreak \\
		\end{tabular}
		\\
		\cmidrule{2-3}
		keyEqualsValuesBraces\newline Brackets &
		\begin{lstlisting}[style=tcblatex,escapeinside={(*@}{@*)},morekeywords={key},nolol=true,]
before words(*@$\BeginStartsOnOwnLine$@*)
key(*@$\EqualsStartsOnOwnLine$@*)=(*@$\BodyStartsOnOwnLine$@*)
(*@$\langle$\itshape{braces/brackets}$\rangle$@*)
  \end{lstlisting}
		                                       &
		\begin{tabular}[t]{c@{~}l@{}}
			$\BeginStartsOnOwnLine$  & KeyStartsOnOwnLine          \\
			$\EqualsStartsOnOwnLine$ & EqualsStartsOnOwnLine       \\
			$\BodyStartsOnOwnLine$   & EqualsFinishesWithLineBreak \\
		\end{tabular}
		\\
		\cmidrule{2-3}
		items                                  &
		\begin{lstlisting}[style=tcblatex,escapeinside={(*@}{@*)},nolol=true]
before words(*@$\BeginStartsOnOwnLine$@*)
\item(*@$\BodyStartsOnOwnLine$@*)
...
  \end{lstlisting}
		                                       &
		\begin{tabular}[t]{c@{~}l@{}}
			$\BeginStartsOnOwnLine$ & ItemStartsOnOwnLine       \\
			$\BodyStartsOnOwnLine$  & ItemFinishesWithLineBreak \\
		\end{tabular}
		\\
		\cmidrule{2-3}
		specialBeginEnd                        &
		\begin{lstlisting}[style=tcblatex,escapeinside={(*@}{@*)},nolol=true]
before words(*@$\BeginStartsOnOwnLine$@*)
\[(*@$\BodyStartsOnOwnLine$@*)
body of special(*@$\EndStartsOnOwnLine$@*)
\](*@$\EndFinishesWithLineBreak$@*)
after words
  \end{lstlisting}
		                                       &
		\begin{tabular}[t]{c@{~}l@{}}
			$\BeginStartsOnOwnLine$     & SpecialBeginStartsOnOwnLine     \\
			$\BodyStartsOnOwnLine$      & SpecialBodyStartsOnOwnLine      \\
			$\EndStartsOnOwnLine$       & SpecialEndStartsOnOwnLine       \\
			$\EndFinishesWithLineBreak$ & SpecialEndFinishesWithLineBreak \\
		\end{tabular}
		\\
		\bottomrule
	\end{longtable}
