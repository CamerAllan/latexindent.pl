A distinção entre conteúdo \emph{real} e
  conteúdo \emph{intencional} está
  relacionada, ainda, à distinção entre o
  conceito husserliano de \emph{experiência} e
  o uso popular desse termo.
No sentido comum, o \term{experimentado} é
  um complexo de eventos exteriores, e o
  \term{experimentar} consiste em percepções
  (além de julgamentos e outros atos) nas
  quais tais eventos aparecem como objetos, e
  objetos frequentemente relacionados ao ego
  empírico.
Nesse sentido, diz-se, por exemplo, que se
  \term{experimentou} uma guerra.
No sentido fenomenológico, no entanto, é
  evidente que os eventos ou objetos externos
  não estão dentro do ego que os experimenta,
  nem são seu conteúdo ou suas partes
  constituintes \cite[5.][3]{lu}.
Experimentar eventos exteriores, nesse
  sentido, significa direcionar certos atos de
  percepção a tais eventos, de modo que certos
  conteúdos constituem, então, uma unidade de
  consciência no fluxo unificado de um ego
  empírico.
Nesse caso, temos um todo \emph{real} do
  qual se pode dizer que cada parte é de fato
  \emph{experimentada}.
Enquanto no primeiro sentido há uma
  distinção entre o conteúdo da consciência e
  aquilo que é experimentado (e.g.\, entre a
  sensação e aquilo que é sentido), nesse
  último sentido aquilo que o ego ou a
  consciência experimenta \emph{é} seu
  conteúdo.
